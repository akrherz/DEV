\documentclass{article}
\usepackage{fullpage}
\usepackage{fancyhdr}
\pagestyle{fancy}
\fancyhf{}
\fancyhead[LE,RO]{\bfseries\thepage}
\fancyhead[LO,RE]{\bfseries\leftmark}
\fancypagestyle{plain}{
  \fancyhead{}
  \renewcommand{\headrulewidth}{0pt}
}

% Lots of short hand equations

\newcommand{\ddt}{\frac{\delta}{\delta t}}
\newcommand{\ddpsi}{\frac{\delta}{\delta \psi}}
\newcommand{\ddphi}{\frac{\delta}{\delta \phi}}

\newcommand{\dpsidp}{\frac{\delta \psi}{\delta p}}
\newcommand{\dpsidphi}{\frac{\delta \psi}{\delta \phi}}

\newcommand{\dozdp}{\frac{\delta \omega_z}{\delta p}}
\newcommand{\dozdphi}{\frac{\delta \omega_z}{\delta \phi}}

\newcommand{\dudt}{\frac{\delta u}{\delta t}}
\newcommand{\duzdt}{\frac{\delta u_z}{\delta t}}
\newcommand{\duzdp}{\frac{\delta u_z}{\delta p}}
\newcommand{\duzdphi}{\frac{\delta u_z}{\delta \phi}}
\newcommand{\dudl}{\frac{\delta u}{\delta \lambda}}

\newcommand{\dvdt}{\frac{\delta v}{\delta t}}
\newcommand{\dvzdp}{\frac{\delta v_z}{\delta p}}
\newcommand{\dvzdphi}{\frac{\delta v_z}{\delta \phi}}
\newcommand{\dvzdt}{\frac{\delta v_z}{\delta t}}
\newcommand{\dvdl}{\frac{\delta v}{\delta \lambda}}
\newcommand{\dvdp}{\frac{\delta v}{\delta p}}

\newcommand{\ddp}{\frac{\delta}{\delta p}}
\newcommand{\dtdt}{\frac{\delta \theta}{\delta t}}
\newcommand{\dtzdt}{\frac{\delta \theta_z}{\delta t}}
\newcommand{\dtzdp}{\frac{\delta \theta_z}{\delta p}}
\newcommand{\dtzdphi}{\frac{\delta \theta_z}{\delta \phi}}

\newcommand{\dZzdp}{\frac{\delta z_z}{\delta p}}

\newcommand{\dTzdphi}{\frac{\delta T_z}{\delta \phi}}
\newcommand{\dTzdt}{\frac{\delta T_z}{\delta t}}

\newcommand{\ooacsp}{\frac{1}{a\cos^2{\phi}}}
\newcommand{\ooascp}{\frac{1}{a^2 \cos{\phi}}}
\newcommand{\ooacp}{\frac{1}{a\cos{\phi}}}
\newcommand{\focp}{\frac{f}{\cos{\phi}}}

\newcommand{\oog}{\frac{1}{g}}
\newcommand{\oots}{\frac{1}{2\overline{\sigma}}}

% Integrals 
\newcommand{\iopo}{\int_{0}^{p_0}}
\newcommand{\ip}{\int_{-\frac{\pi}{2}}^{\frac{\pi}{2}}}
\newcommand{\iztp}{\int_{0}^{2\pi}}

% Sumations
\newcommand{\smi}{\sum_{m=-\infty}^{\infty}}
\newcommand{\sni}{\sum_{n=-\infty}^{\infty}}
\newcommand{\snpi}{\sum_{n'=-\infty}^{\infty}}
\newcommand{\snppi}{\sum_{n''=-\infty}^{\infty}}


\begin{document}

Spectral Energetics of the Atmosphere.

\begin{itemize}

\item Saltzman, B. 1957: Equations governing the energetics of the large scales of atmosperic turbulence in the domain of wave number. J. Met, 14, 513-523.

\item Saltzman, B., 1970: Large-scale atmospheric energetics in the wave number domain.  Rev. Geophys. Space Phys, 8, 289-302.

\item Yang, C.-H. 1967: Nonlinear aspects of the large-scale motion in the atmosphere.  Techincal Report 08759-1-T, 173pp.  Univ of Michigan.

\item Steinberg, Win-Nielsen, Yang, 1971: On nonlinear cascades in large-scale atmospheric flow.  J.G.R., 76, 8629-8640.

\end{itemize}

A) Available Potential Energy

\begin{equation}
A = \frac{1}{4 \pi a^2}\oog \iopo \ip \iztp \oots \alpha'^2 (\cos{\phi}) a^2 d\phi d\lambda dp
\label{ape}
\end{equation}

We can define a term $\overline{e}$ to simplify this equation some.

\begin{equation}
\overline{e} = \frac{1}{2\pi} \iztp \oots \alpha'^2 d\lambda
\label{e}
\end{equation}

And so Eqn \ref{ape} reduces to.

\begin{equation}
A = \frac{1}{2g} \iopo \ip \overline{e} (\cos{\phi}) d\phi dp
\end{equation}

Recall that 

\begin{equation}
\alpha' = \sni \alpha_n e^{in\lambda}
\label{alphaprime}
\end{equation}

\begin{equation}
\alpha_n = \frac{1}{2\pi} \iztp \alpha' e^{-in\lambda} d\lambda
\label{alphan}
\end{equation}

So Eqn \ref{e} can be expressed by inserting $\alpha'$ to get

\begin{equation}
\overline{e} = \frac{1}{2\pi} \iztp \oots [\sum_{n} \alpha_n e^{in\lambda} ][\sum_{n'} \alpha_n' e^{in'\lambda} ] d\lambda
\end{equation}

\begin{equation}
\overline{e} = \sni \oots \alpha_n\alpha_{-n} = \oots \alpha_{o}^2 + \sum_{n=1}^{\infty} \frac{1}{\overline{\sigma}} \alpha_n\alpha_{n}^{*}
\end{equation}


\begin{equation}
\frac{\delta \overline{e}}{\delta t} = \sni \oots [ \alpha_{-n} \frac{\delta \alpha_n}{\delta t} + \alpha_n \frac{\delta \alpha_{-n}}{\delta t} ]
\end{equation}

\begin{equation}
\frac{\delta \alpha'}{\delta t} = - (\frac{u}{a \cos{\phi}} \frac{\delta \alpha'}{\delta \lambda} + \frac{v}{a}\frac{\delta \alpha'}{\delta \phi} )
\end{equation}

u and v can be expressed as a sumation of their n components

\begin{equation}
u = \sni U_n e^{in\lambda}
\end{equation}

\begin{equation}
v = \sni V_n e^{in\lambda}
\end{equation}


\begin{equation}
\sni \frac{\delta \alpha_n}{\delta t} e^{in\lambda} = - [ \ooacp \snpi U_{n'} e^{in'\lambda} \snppi (in'') \alpha_{n''} e^{in''\lambda} + \frac{1}{a} \snpi V_{n'} e^{in'\lambda} \snppi \frac{\delta \alpha_{n''}}{\delta \phi} e^{in''\lambda} ]
\end{equation}

\begin{equation}
\sni \frac{\delta \alpha_n}{\delta t} e^{in\lambda} = - \snpi \snppi [ \frac{in''}{a \cos{\phi}} U_{n'} \alpha_{n''} + \frac{1}{a} V_{n'} \frac{\delta \alpha_{n''}}{\delta \phi} ] e^{i(n'+n'')\lambda}
\end{equation}

\begin{equation}
\frac{\delta \alpha_n}{\delta t} = - \frac{1}{2\pi} \iztp [ \snpi \snppi [ \frac{in''}{a \cos{\phi}} U_{n'} \alpha_{n''} + \frac{1}{a} V_{n'} \frac{\delta \alpha_{n''}}{\delta \phi} ] e^{i(n'+n''-n)\lambda} ] d\lambda
\label{dandt}
\end{equation}

Some simple identities include

$n' + n'' -n = 0$

$n'' = m$

$n' = n - m$

So we can then express Eqn \ref{dandt} as

\begin{equation}
\frac{\delta \alpha_n}{\delta t} = - \smi [ \frac{im}{a \cos{\phi}} U_{(n-m)} \alpha_m + \frac{1}{a} V_{(n-m)} \frac{\delta \alpha_m}{\delta \phi} ]
\label{e1}
\end{equation}

\begin{equation}
\frac{\delta \alpha_{-n}}{\delta t} = - \smi [ \frac{im}{a \cos{\phi}} U_{(-n-m)} \alpha_m + \frac{1}{a} V_{(-n-m)} \frac{\delta \alpha_m}{\delta \phi} ]
\label{e2}
\end{equation}

And now we take $\alpha_{-n}$ times Eqn \ref{e1} and $\alpha_n$ times Eqn \ref{e2} and add those two resultant eqns together to get

\begin{equation}
\alpha_{-n} \frac{\delta \alpha_{n}}{\delta t} + \alpha_{n} \frac{\delta \alpha_{-n}}{\delta t} = - \smi [ \frac{im}{a \cos{\phi}} \alpha_m [ U_{(n-m)} \alpha_{-n} + U_{(-n-m)} \alpha_n ] + \frac{1}{a} \frac{\delta \alpha_m}{\delta \phi} [ V_{(n-m)} \alpha_{-n} + V_{(-n-m)} \alpha_n ] ]
\end{equation}

\begin{equation}
\frac{\delta \overline{e}_n}{\delta t} = - \oots \smi [ \frac{im}{a \cos{\phi}} \alpha_m [ U_{(n-m)} \alpha_{-n} + U_{(-n-m)} \alpha_{n} ] + \frac{1}{a} \frac{\delta \alpha_m}{\delta \phi} [ V_{(n-m)} \alpha_{-n} + V_{(-n-m)} \alpha_n ] ]
\end{equation}

Recall that our rate of change of Available Potential Energy (A) is

\begin{equation}
\frac{dA}{dt} = \frac{1}{2g} \iopo \ip \frac{\delta \overline{e}}{\delta t} )\cos{\phi}) \delta \phi \delta p
\end{equation}

\begin{equation}
\frac{dA}{dt} = \frac{1}{2g} \iopo \ip \sni \frac{\delta \overline{e}_n}{\delta t} \cos{\phi} d\phi dp
\end{equation}

\begin{equation}
\frac{dA}{dt} = \sni \frac{d A_n}{d t}
\end{equation}

where

\begin{equation}
\frac{dA_n}{dt} = \frac{1}{2g} \iztp \ip \frac{\delta \overline{e}_n}{\delta t} (\cos{\phi}) du dp
\end{equation}

B) Kinectic energy equation

\begin{equation}
K = \frac{1}{4\pi a^2} \frac{1}{g} \iopo \ip \iztp \frac{1}{2} \nabla \psi \nabla \psi (a^2 \cos{\phi}) d\phi d\alpha dp
\end{equation}

\begin{equation}
\frac{dK}{dt} = \frac{1}{4\pi} \frac{1}{g} \iopo \ip \iztp \nabla \psi \frac{\delta}{\delta t} (\nabla \psi) \cos{\phi} d\phi d\lambda dp
\end{equation}

\begin{equation}
\frac{dK}{dt} = - \frac{1}{4\pi} \frac{1}{g} \iopo \ip \iztp \psi \frac{\delta \zeta}{\delta t} (\cos{\phi}) d\phi d\lambda dp
\end{equation}

\begin{equation}
\psi = \sni \Psi_n e^{in\lambda}
\end{equation}

\begin{equation}
\Psi_n = \frac{1}{2\pi} \iztp \psi e^{-in\lambda} d\lambda
\end{equation}

\begin{equation}
\zeta = \frac{1}{a \cos{\phi}} [ \frac{\delta v}{\delta \lambda} - \frac{\delta}{\delta \phi} (u \cos{\phi}) ]
\end{equation}

\begin{equation}
\zeta = \frac{1}{a \cos{\phi}} [ \frac{\delta}{\delta \lambda} (\frac{\delta \psi}{a \cos{\phi} \delta \lambda} + \frac{\delta}{\delta \phi} ( \cos{\phi} \frac{\delta \psi}{a \delta \phi} ) ]
\end{equation}

\begin{equation}
\zeta = \frac{1}{a^2 \cos^2{\phi}} \frac{\delta^2 \psi}{\delta \lambda^2} + \ooascp \frac{\delta}{\delta \phi} ( \cos{\phi} \frac{\delta \psi}{\delta \phi} )
\end{equation}

\begin{equation}
\zeta = \sni [ - \frac{n^2}{a^2 \cos^2{\phi}} \Psi_n + \ooascp \ddphi (\cos{\phi} \frac{\Psi_n}{\delta \phi} ) ] e^{in\lambda}
\end{equation}

\begin{equation}
\zeta = \sni Z_n e^{in\lambda}
\end{equation}

\begin{equation}
Z_n = - \frac{n^2}{a^2 \cos^2{\phi}} \Psi_n + \frac{1}{a^2 \cos{\phi}} \frac{\delta}{\delta \phi} ( \cos{\phi} \frac{\delta \Psi_n}{\delta \psi} )
\end{equation}

\begin{equation}
\frac{\delta \zeta}{\delta t} = - \vec{V} \nabla \zeta = \frac{1}{a^2 \cos{\phi}} \frac{\delta \psi}{\delta \phi} \frac{\delta \zeta}{\delta \lambda} - \ooascp \frac{\delta \psi}{\delta \lambda} \frac{\delta \zeta}{\delta \phi} 
\end{equation}

\begin{equation}
= \ooascp (\frac{\delta \zeta}{\delta \lambda} \frac{\delta \psi}{\delta \phi} - \frac{\delta \zeta}{\delta \phi} \frac{\delta \psi}{\delta \alpha} )
\end{equation}

\begin{equation}
\frac{dK}{dt} = - \frac{1}{4\pi} \frac{1}{g} \iopo  \ip \iztp [ \sum_{n'} \Psi_{n'} e^{in'\lambda} ] [ \sum_{n''} \frac{\delta Z_{n''}}{\delta t} e^{in''\lambda} ] (\cos{\phi}) d\lambda d\phi dp
\end{equation}

\begin{equation}
\frac{dK}{dt} = - \frac{1}{4\pi} \frac{1}{g} \iopo \ip \iztp \sum_{n'} \sum_{n''} \Psi_{n'} \frac{\delta Z_{n''}}{\delta t} e^{i(n'+n'')\lambda} (\cos{\phi}) d\lambda d\phi dp
\end{equation}

\begin{equation}
\frac{dK}{dt} = - \frac{1}{2g} \iopo \ip \sum_n [ \Psi_{n} \frac{\delta Z_{-n}}{\delta t} + \Psi_{-n} \frac{\delta Z_n}{\delta t} ] \cos{\phi} d\phi dp
\end{equation}

\begin{equation}
\sni \frac{\delta Z_n}{\delta t} e^{in\lambda} = \ooascp  [ \sum_{n'}  Z_{n'} (in') e^{in'\lambda} \sum_{n''} \frac{\delta \Psi_{n''}}{\delta \phi} e^{in''\lambda} - \sum_{n'} \frac{\delta Z_{n'}}{\delta \phi} e^{in'\lambda} \sum_{n''} \Psi_{n''} (in'') e^{in''\lambda} ]
\end{equation}

\begin{equation}
= \ooascp \sum_{n'} \sum_{n''} [ (in') Z_{n'} \frac{\delta \Psi_{n''}}{\delta \phi} - (in'') \Psi_{n''} \frac{\delta Z_{n''}}{\delta \phi} ] e^{i(n'+n'')\lambda}
\end{equation}

Recall again that

$n'+n''-n = 0$

$n'' = m$

$n'= n-m$

\begin{equation}
\frac{\delta Z_n}{\delta t} = \sum_m [ \frac{i(n-m)}{a^2 \cos{\phi}} Z_{n-m} \frac{\delta \Psi_m}{\delta \phi} - \frac{im}{a^2 \cos{\phi}} \Psi_m \frac{\delta Z_{n-m}}{\delta \phi} ]
\label{dzndt}
\end{equation}

\begin{equation}
\frac{\delta Z_{-n}}{\delta t} = \sum_m [ \frac{i(-n-m)}{a^2 \cos{\phi}} Z_{-n-m} \frac{\delta \Psi_m}{\delta \phi} - \frac{im}{a^2 \cos{\phi}} \Psi_m \frac{\delta Z_{-n-m}}{\delta \phi} ]
\label{dzmndt}
\end{equation}

To get an expression of the rate of change of Kinetic Energy, we take $\psi_{-n}$ times Eqn \ref{dzndt} and $\psi_n$ times Eqn \ref{dzmndt}.  We get

%split
\begin{equation}
\frac{dK}{dt} = - \frac{1}{2g} \iopo \ip \sum_n [ \sum_m \frac{1}{a^2} \frac{\delta \Psi_m}{\delta \phi} [ i(n-m) Z_{n-m} \Psi_{-n} + i(-n-m) Z_{-n-m} \Psi_n ] - \frac{1}{a^2} im \Psi_m [ \Psi_{-n} \frac{\delta Z_{n-m}}{\delta \phi} + \Psi_n \frac{\delta Z_{-n-m}}{\delta \phi}] ] d\phi dp
\label{dKdt}
\end{equation}

We can pull the $\sum_n$ out front of Eqn \ref{dKdt}

\begin{equation}
\frac{dK}{dt} = \sum_n [ - \frac{1}{2g} \iopo \ip \sum_m d\phi dp
\end{equation}

Also...

\begin{equation}
\frac{dK}{dt} = \sum_n \frac{dK_n}{dt}
\end{equation}

% split
\begin{equation}
\frac{dK_n}{dt} = - \frac{1}{2g} \iopo \ip \sum_m [ \frac{\delta \Psi_m}{a^2 \delta \phi} [i(n-m) Z_{n-m} \Psi_{-n} + i(-n-m) Z_{-n-m} \Psi_{n} ] - \frac{im}{a^2} \Psi_m [ \Psi_{-n} \frac{\delta Z_{n-m}}{\delta \phi} + \Psi_n \frac{\delta Z_{-n-m}}{\delta \phi} ] ] d\phi dp
\end{equation}

C. Enstrophy

Recall that our barotropic vorticity equation is $\frac{\delta \zeta}{\delta t} + \vec{V} \nabla {\zeta} = 0$ and we re-express as

\begin{equation}
\frac{\delta}{\delta t} (\frac{1}{2} \zeta^2 ) = - \vec{V} \nabla (\frac{1}{2} \zeta^2 )
\end{equation}

We can express Enstrophy (E) as 

\begin{equation}
E = \frac{1}{4\pi a^2} \frac{1}{p_o} \iopo \ip \iztp \frac{1}{2} \zeta^2 a^2 \cos{\phi} d\phi d\alpha dp
\end{equation}

and the rate of change is 

\begin{equation}
\frac{dE}{dt} = \frac{1}{4\pi} \frac{1}{p_o} \iopo \ip \iztp \zeta \frac{\delta \zeta}{\delta t} \cos{\phi} d\phi d\alpha dp
\end{equation}

\begin{equation}
= \frac{1}{2 p_o} \iopo \ip {(\zeta \frac{\delta \zeta}{\delta t})}_z \cos{\phi} d\phi dp
\end{equation}

\begin{equation}
(\zeta \frac{\delta \zeta}{\delta t})_z = \frac{1}{2\pi} \iztp [ Z_{n'} \zeta_{n'} e^{in'\lambda} ] [ \sum_{n''} \frac{\delta Z_{n''}}{\delta t} e^{in''\lambda} ] d\lambda
\end{equation}

\begin{equation}
(\zeta \frac{\delta \zeta}{\delta t})_z = \frac{1}{2\pi} \iztp \sum_{n'} \sum_{n''} Z_{n'} \frac{\delta Z_{n''}}{\delta t} e^{i(n'+n'')\lambda} d\alpha
\end{equation}

\begin{equation}
(\zeta \frac{\delta \zeta}{\delta t})_z = \sum_n (Z_n \frac{Z_{-n}}{\delta t} + Z_{-n} \frac{\delta Z_n}{\delta t} )
\end{equation}

% split
\begin{equation}
Z_n \frac{\delta Z_{-n}}{\delta t} + Z_{-n} \frac{\delta Z_n}{\delta t} = \sum_m [ \frac{i(n-m)}{a^2 \cos{\phi}} Z_n Z_{n-m} \frac{\delta \Psi_m}{\delta \phi} - \frac{im}{a^2 \cos{\phi}} Z_n \Psi_m \frac{\delta Z_{n-m}}{\delta \phi} + \frac{i(-n-m)}{a^2 \cos{\phi}} Z_{-n} Z_{-n-m} \frac{\delta \Psi_m}{\delta \phi} - \frac{im}{a^2 \cos{\phi}} Z_{-n} \Psi_m \frac{\delta Z_{-n-m}}{\delta \phi} ]
\end{equation}

\begin{equation}
\frac{dE}{dt} = \frac{1}{2g} \iztp \ip \sum_n [ Z_n \frac{\delta Z_{-n}}{\delta t} + Z_{-n} \frac{\delta Z_n}{\delta t} ] (\cos{\phi}) d\phi dp
\end{equation}

\begin{equation}
\frac{dE}{dt} = \sum_n [ \frac{1}{2g} \iztp \ip (Z_n \frac{\delta Z_{-n}}{\delta t} + Z_{-n} \frac{\delta Z_n}{\delta t} ) (\cos{\phi}) d\phi dp ]
\end{equation}

\begin{equation}
\frac{dE}{dt} = \sum_n \frac{dE_n}{dt}
\end{equation}

\begin{equation}
\frac{dE_n}{dt} = \frac{1}{2g} \iztp \ip [ \sum_m \frac{\delta \Psi_m}{\delta \phi} [ \frac{i(n-m)}{a^2} Z_n Z_{n-m} + \frac{1(-n-m)}{a^2} Z_{-n} Z_{-n-m} ] - \frac{im}{a^2} \Psi_m [ Z_n \frac{\delta Z_{n-m}}{\delta \phi} + Z_{-n} \frac{\delta Z_{-n-m}}{\delta \phi} ] ] d\phi dp
\end{equation}

Diagnostic study:

1. Wiin-Nielsen (1967, Tellus)

$8 \leq n \leq 15$

$A \sim n^{-3}$

$K \sim n^{-3}$

2. Steinerg et al (1971, JGR)

$E \sim n^{-1}$

(i) nonlinear transfer of K, A, and E

\begin{equation}
\frac{\delta A(m)}{\delta t} \sim C_{A}(m|n,l) + C_{A}(m|0) + F_{A}(m)
\end{equation}
\begin{equation}
\frac{\delta K(m)}{\delta t} \sim C_{K}(m|n,l) + C_{K}(m|0) + F_{K}(m)
\end{equation}
\begin{equation}
\frac{\delta E(m)}{\delta t} \sim C_{E}(m|n,l) + C_{E}(m|0) + F_{E}(m) + \beta (m)
\end{equation}

(ii) Flux of K and E:

\begin{equation}
\frac{\delta F}{\delta m} = - C_{K}(m|n,l)
\end{equation}

\begin{equation}
\frac{\delta H}{\delta m} = - C_{E}(m|n,l)
\end{equation}


\end{document}
