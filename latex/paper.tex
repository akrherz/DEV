\documentclass{article}

\usepackage{graphicx}
\usepackage{multicol}
\usepackage{floatflt}
\usepackage{fullpage}

\include{myDefs}


\title{Kain-Fritsch Scheme in the Workstation ETA: A Case Study}
\author{Daryl Herzmann}
\date{17 April 2001}

\begin{document}



\maketitle

\begin{abstract}
A versatile version of the workstation ETA (hereafter WSETA) model is used to simulate a problematic model integration for 04 August 2000.  The original WSETA run produced meso-$\beta$ areas of excessive precipitation.  From an initial analysis of the output data, it was seen that the model was not producing a valid solution.  A possible cause for these problems was assumed to be an unactivated Kain-Fritsch cumulus parameterization scheme (hereafter CPS).  A model rerun was completed using a 400 hPa search layer for initiation of the scheme, instead of a 200 hPa layer.  The results of the rerun were encouraging, but did not fully resolve the problem.

\end{abstract}

\Line

%
\begin{figure*}[bt]
\begin{center}
    \resizebox{.60\columnwidth}{!}{\includegraphics[angle=-90]{PLOTS/00080400_f36_36prec.eps}}
\end{center}
\label{firstPrec}
 \caption{04 August 2000 0000F036 UTC WSETA -- Plot shows 36 h accumulated precipitation.  Contours every 40 mm. Values of over 280mm are shown in South Dakota.}
\end{figure*}

% You appear on page two at the top
\begin{figure*}[t]
\begin{center}
\begin{minipage}{.45\textwidth}
    \resizebox{\columnwidth}{!}{\includegraphics[angle=-90]{PLOTS/00080400_f30_gridP.eps}}
\end{minipage}
\hspace{14pt}
\begin{minipage}{.45\textwidth}
    \resizebox{\columnwidth}{!}{\includegraphics[angle=-90]{PLOTS/00080400_f30_gridC.eps}}
\end{minipage}
\end{center}
\caption{04 August 2000 0000F030 UTC WSETA -- Right plot shows total precipitation for each grid point. 
        The left plot shows the amount generated by the cumulus scheme.  Amounts are in mm.  LATS and LONGS are plotted for a frame
 of reference.}
\end{figure*}

%appear on page two at the bottom
\begin{figure*}[b]
\begin{center}
    \resizebox{.8\columnwidth}{!}{\includegraphics[angle=-90]{PLOTS/00080400_f29_f30_PWTR.eps}}
\end{center}
 \caption{04 August 2000 0000F030 UTC WSETA -- 1 h change in precipitable water from F29 to F30.  Values are given in mm .}
\end{figure*}

%
\begin{figure*}[bt]
\begin{center}
    \resizebox{.60\columnwidth}{!}{\includegraphics[angle=-90]{PLOTS/00080400_f18_1h.eps}}
\end{center}
\label{firstPrec}
 \caption{04 August 2000 0000F018 UTC WSETA -- Plot shows 1 h accumulated precipitation.  Contours every 3 mm. A bullseye target of 12 mm can be seen in Southeast South Dakota.}
\end{figure*}

%
\begin{figure*}[bt]
\begin{center}
    \resizebox{.60\columnwidth}{!}{\includegraphics[angle=-90]{PLOTS/00080400_f24_1h.eps}}
\end{center}
\label{firstPrec}
 \caption{04 August 2000 0000F024 UTC WSETA -- Plot shows 1 h accumulated precipitation.  Contours every 3 mm. Two bulleye targets can be seen in Central South Dakota.}
\end{figure*}

%
\begin{figure*}[bt]
\begin{center}
    \resizebox{.60\columnwidth}{!}{\includegraphics[angle=-90]{PLOTS/00080400_f30_1h.eps}}
\end{center}
\label{firstPrec}
 \caption{04 August 2000 0000F030 UTC WSETA -- Plot shows 1 h accumulated precipitation.  Contours every 5 mm.}
\end{figure*}

%
\begin{figure*}[bt]
\begin{center}
\begin{minipage}{.45\textwidth}
    \resizebox{.90\columnwidth}{!}{\includegraphics[angle=-90]{PLOTS/00080400_f33_1h.eps}}
\end{minipage}
\hspace{14pt}
\begin{minipage}{.45\textwidth}
    \resizebox{.90\columnwidth}{!}{\includegraphics[angle=-90]{PLOTS/00080400_f36_1h.eps}}
\end{minipage}\\

\begin{minipage}{.45\textwidth}
    \resizebox{.90\columnwidth}{!}{\includegraphics[angle=-90]{PLOTS/00080400_f33_1h_CPS.eps}}
\end{minipage}
\hspace{14pt}
\begin{minipage}{.45\textwidth}
    \resizebox{.90\columnwidth}{!}{\includegraphics[angle=-90]{PLOTS/00080400_f36_1h_CPS.eps}}
\end{minipage}\\

\end{center}
 \caption{04 August 2000 0000 UTC WSETA -- Top Row: Left plot shows F33 1h precipitation and the right plot shows F36 1h precipitation.  Bottom Row:  Left plot shows F33 1h precipitation generated from the cumulus scheme and the left plot shows the same for F36.  Note that very little of the total precipitation was generated from the CPS.  Values are contoured every 5 mm.}
\end{figure*}

%
\begin{figure*}[bt]
\begin{center}
    \resizebox{.60\columnwidth}{!}{\includegraphics[angle=-90]{PLOTS/00080400_f18_1h_400.eps}}
\end{center}
\label{firstPrec}
 \caption{04 August 2000 0000F018 UTC WSETA -- Plot shows 1 h accumulated precipitation.  Contours every 3 mm.}
\end{figure*}

%
\begin{figure*}[bt]
\begin{center}
    \resizebox{.90\columnwidth}{!}{\includegraphics[angle=-90]{PLOTS/00080400_f24_1h_400.eps}}
\end{center}
\label{firstPrec}
 \caption{04 August 2000 0000F024 UTC WSETA -- Plot shows 1 h accumulated precipitation.  Contours every 3 mm.}
\end{figure*}

%
\begin{figure*}[bt]
\begin{center}
    \resizebox{.90\columnwidth}{!}{\includegraphics[angle=-90]{PLOTS/00080400_f30_1h_400.eps}}
\end{center}
\label{firstPrec}
 \caption{04 August 2000 0000F030 UTC WSETA -- Plot shows 1 h accumulated precipitation.  Contours every 5 mm.}
\end{figure*}


%
\begin{figure*}[bt]
\begin{center}
\begin{minipage}{.45\textwidth}
    \resizebox{.90\columnwidth}{!}{\includegraphics[angle=-90]{PLOTS/00080400_f33_1h_400.eps}}
\end{minipage}
\hspace{14pt}
\begin{minipage}{.45\textwidth}
    \resizebox{.90\columnwidth}{!}{\includegraphics[angle=-90]{PLOTS/00080400_f36_1h_400.eps}}
\end{minipage}\\

\begin{minipage}{.45\textwidth}
    \resizebox{.90\columnwidth}{!}{\includegraphics[angle=-90]{PLOTS/00080400_f33_1h_400_CPS.eps}}
\end{minipage}
\hspace{14pt}
\begin{minipage}{.45\textwidth}
    \resizebox{.90\columnwidth}{!}{\includegraphics[angle=-90]{PLOTS/00080400_f36_1h_400_CPS.eps}}
\end{minipage}\\

\end{center}
 \caption{04 August 2000 0000 UTC WSETA -- Top Row: Left plot shows F33 1h precipitation and the right plot shows F36 1h precipitation.  Bottom Row:  Left plot shows F33 1h precipitation generated from the cumulus scheme and the left plot shows the same for F36. Values are contoured every 5 mm.}
\end{figure*}

\begin{multicols}{2}

\section{Introduction}
With the recent exponential increase in affordable computing power, running limited-area numerical
weather models has become practical for research and operational use by a wide variety of institutions
(Warner \etal 1997).  Such limited-area models are capable of running at higher resolutions and with different physics packages than those currently used operationally at the National Centre for Environmental Prediction.  The possibilities of these versatile models are very exciting.

A version of the WSETA has been made available for use by Scientific Operation Officers (SOOs) at the National Weather Service Weather Forecast Offices (WFOs).  Front-end scripts for the model make running the WSETA operationally, trivial.  This model, with small modifications, has been running in real-time at Iowa State for the past year.  Output is currently analyzed and made available via the Internet (see note 1).

During the WSETA use at Iowa State, a run from 04 August 2000 0000 UTC was initially noticed for its extreme production of precipitation over a meso-$\beta$ area in Eastern South Dakota. Figure 1 shows 36 h precipitation values of over 280 mm located in a bullseye area.  This amount seemed remarkable at the time and further investigation was needed.

This case was analyzed by Daryl Herzmann for his senior thesis.  It was orginally assumed that the excessive precipitation was caused by lateral boundry conditons incorrectly advected in during the model integration.  This hypothesis was partially invalidated from the results that moving the model domain around did not remove the grid point storm.  While problems were noted with lateral boundary conditions, it was determined that the errors were not causing the grid point storm (Herzmann 2000).

This investigation looks to explain why the excessive precipitation is produced from the WSETA model.  The hypothesis is that the convective scheme did not activate because the convection was elevated and thus not handled by the convective scheme as orginally configured.  The specifics to why the scheme did not active will be covered later.  The goal of this investigation was to reconfigure the model to better handle this case, meaning a more realistic precipitation prediction would be produced.

\section{Kain-Fritcsh Scheme in the WSETA}
The WSETA has the ability to utilize the Kain-Fritcsh cumulus parameterization scheme (hereafter KF) in replace of the Betts-Miller-Janjic scheme (hereafter BMJ).  This makes the WSETA very attractive to use in convective summer events, since it will often produce signifigantly different results than the operational NCEP ETA, which uses the BMJ.

The KF implementation is based on the Fritsch-Chappell scheme detailed in Fritsch \etal 1993.  The purpose of the scheme is to resolve sub-grid scale, convective elements in numerical models.  The KF was designed to be very rigorous in its conservation of mass, thermal energy, total moisture, and momentum properties (Kain (\etal ????).  The scheme works by allowing ...

\section{Model Configuration}
The WSETA was configured to run on a 75 east-west by 121 north-south horizontal grid with 45 $\eta$ levels. With a rough grid spacing of 10 km, the domain covers an area roughly six times the size of Iowa.  The domain was centered at 44 \deg N and -97 \deg W.  This setup was not by chance, it was selected to be a size useful for operation use at the WFO in Des Moines, Iowa.  A fundamental model timestemp of 20 s was chosen and the KF was used.  The rest of the parameters were left un-touched from the base WSETA.

Prior runs had been made varying the domain centering and the cumulus parameterization used.  Some of those runs are mentioned in this paper.  The configurations for those runs were identical to the aforementioned, but one setting was usually varied, ie scheme or domain centering.

\section{Initial Run}
There was clearly something wrong with the initial run of the WSETA for 04 August 2000.  The generated area of total precipitation in a bullseye form generated speculation on its origin.  The synoptic conditions are documented in Herzmann 2000.  A surface low pressure system was developing over Eastern Colorado and was moving into the model domain from the southwest.  The upper level forcing was weak with a broad ridge centered over the western section of the model domain.  Surface temperatures had warmed signicantly during the day (F12-F24), with a strong southerly wind advecting 70+ \degf dewpoints into Eastern South Dakota.

Precipitation begins to break out in the model at forecast hour 15 (10 AM LDT).  By F18, the model was producing a broad area of extremely light precipitation.  Figure 2 shows this area, note the bullseye area of 1 cm \hr rainfall.  The area slides off to the southeast and dissipates by F24.  Figure 3 shows the 1 h precipitation values valid at F24.  At this hour, two unique areas of intense rainfall are shown.  At this time, the areas are by no means unrealistic.  At F30, the two areas on precipitation have moved different directions.  Figure 4 shows 1 h precipitation at F30, note that the southern most "target" from F24 has moved to the southeast, while the "target" to the north has moved almost due east and intensified.

At F30, an intense area of precipitation exists over extreme Eastern South Dakota.  Figure 5 shows rainfall values at F30 and then at F33.  It is interesting to note what the precipitation area does during this time, it illustrates a classic supercell spliting!  The right figure in figure 5 shows the right turning right-flank storm and the faster left-flank storm racing to the north.  Given that this storm is obviously powerful, one would expect the cumulus scheme to have activated to resolve this storm.  The lower plots on figure 5 shows that very little of the generated rainfall was a result of the CPS.  Analysis of other features show intense generation of gravity waves from this storm.

Looking at the grid point values for precipital water and precipitation during the model integration also suggest that the model was having problems.  Figure 2 shows the grid point values of model generated precipitation between F29 and F30 for a region in Central South Dakota.  The right most plot of Figure 2 shows the total precipitation, while the right plot shows the portion generated from the cumulus scheme.  This intense area of precipitation would imply descent, if not rigorous convection; the area is certainly not evidence of a statiform event.  Figure 3 shows the one hour change in precipitable water between F29 and F30, which relates to the precipitation data shown in F
igure 2.   Figure 3 raises some interesting questions about what the model was resolving.  Considering a simple relationship between precipitation and advection of moisture into a column and we ignore evaporation for this s
hort time span, we have

\begin{equation}
 S = P - \Delta Q = \frac{\delta W}{\delta t}
 \label{Simple Water Conservation}
\end{equation} 
Here the change in column integrated water, W, which I relate to storage, S, is balanced by precipitation out (negative contributio
n) and moisture divergence (negative contribution).  We would assume, if precipitable water is conserved, that any imbalance betwee
n divergence and precipitation would be balanced by a loss or gain in the storage term (precipitable water).  Making this assumptio
n, we note in Figure 2 that an example grid point had a value of 3 mm for change in precipitable water, while generating 52 mm of p
recipitation.  This would equate to 55 mm of moisture divergence during that hour! Considering that the convective timescale is 30 
m, one could imagine two cycles of lift and precipitation through a column, but I would then assume the column experienced rigorous
 convection and the KF scheme should have activated.


Clearly, one could envision such a situation taking place with supercells breaking out along a warm front, but the small bullseye targets seem unlikely with a situation of statiform over-running.


\section{Model Reconfiguration}
One of the goals of this project was to create a WSETA run that would generate a more realistic solution.  It was hoped that this would be accomplished by "turning up" the sensitivity for the scheme to activate during the model run.  While surveying the FORTRAN source code, it was interesting to note that the scheme was only "looking" at the lowest 200 hPa of atmosphere for activation triggers.  It was also interesting to note that the code contained commented sections that would cause the model to "look" in the lowest 400 hPa.  The code was thus modified this way and it compiled successfully.  The modifications were made in the kfdrive.f file.

\section{Model Comparison}
By changing the depth from which the CPS searches to activate, the model's output was subtly changed in some locations, while marketably changed in others.  The original run began to have problems allready at F18, with a rather intense area of precip moving to the southeast (Figure 2).  Figure 6 shows the same time, but for the reconfigured run.  The most noticable feature of Figure 6 is the area of heavier precipitation is smeared out compared to Figure 2.  This would suggest that the model is doing a better job of resolving this situation.

Continuing on to F24, Figure 7 shows that the model has still avoided the grid point storms with the bullseye precipitation features.  Figure 7 compares to Figure 3.  It is remarkable the differences between the plots. The model fires convection further to the west.  The 1 h precipitation values are still reasonable, with the highest amounts nearing 1 cm \hr.

By F30, the rerun is avoiding the bullseye areas of precipitation that were thrown in the original run.  Figure 8 shows the 1 h rainfall valid at F30.  This plots compares to Figure 4.  A noticable improvement is noted in the rerun, with less intense areas of precipitation.  Areas of 15 mm are noted in the rerun, while areas of 20 mm and 60 mm can be seen in Figure 4. 

Up until F33, the rerun had been successful at not producing the grid point storms.  Figure 9 unfortunately shows the model blowing up precipitation in confined areas.  The formation evident at F36 seems peculiar.  The west element would appear to be a sqall line, but the rational for the east element is not understood.  An important factor to consider when a limited area model is integrated out to 36 hours and beyond is to consider the effects of lateral boundaries.  Errors in advecting systems into the model domain, can have degrading effects on the model solution (Warner \etal 1997).
The resulting confidence in F33 solutions and beyond are questionable.

\section{Conclusions}
The results of the model rerun are certainly encouraging.  Up until F33, the model was producing more realistic solutions by having the CPS activate eariler before large CAPE values built up in the afternoon hours.  The resultant storms prodcued less intense rainfall rates and the signatures of the output precipitation areas were more realistic.

The problems shown in Figure 9 from the model rerun are discouraging.  Comparing Figure 9 with Figure 5, the models solution needs to be throughly questioned.

\section{Notes}
\textbf{1}  Output from the WSETA run at Iowa State can be viewed at http://www.meteor.iastate.edu/~iowaeta .  At the time of this publication, the WSETA was being run for 0000 UTC and 0600 UTC initializations.

\textbf{2}  The observed amount of precipitation for this event was heavy in some areas.  Portions of SE South Dakota received over 3 inches of rain.  Northern Iowa also recieved 1 and 2 inch rainfalls the evening of 4 August 2000 and the morning of 5 August 2000.

\textbf{3}  This case was also ran using the BMJ CPS.  The precipitation patterns were lacking the bullseye type patterns, but the model underestimated the amount of precipitation that verified.  While the KF produced wild results, it at least recognized a major precipitation event.

\section{References}

\ref Black, T., 1994:  The new NMC mesoscale Eta model:  description and forecast examples.  \wf \vol 9 265-278.

\ref Colle B. A., K.J. Westrick, and C.F. Mass, 1994:  Evolution of MM5 and Eta-10 Precipitation Forecasts over the Pacific Northwest during the Cool Season.  \wf \vol 14 137-154

\ref Herzmann, Daryl E., 2000: Effects of Domain Centering with the Workstation ETA.  Presented at the Iowa State Undergraduation Research Seminar.

\ref Janjic, Z. I., 1984:  Non-linear advection schemes and energy cascade on semi-staggered grids. {\it Mon. Wea. Rev.}, \vol 112 1234-1245.

\ref Kain, J.S., and J.M. Fritsch, 1992:  The role of the convective "trigger function" in numerical forecasts of mesoscale convective systems.  {\it Meteor. Atmos. Phys.}, \vol 49 93-106.

\ref -----, and ----, 1993.  Convective parameterization for mesoscale models:  The Kain-Fritsch scheme.  {\it The representation of cumulus convection in numerical models}, American Meteorological Society, 165-170..

\ref Rogers, E., D. G. Deaven, and G. J. DiMego, 1995:  The regional analysis system for the operational Eta model:  Original 80 km configuration, recent changes, and future plans. 
 \wf \vol 10 810-825.

\ref -----, Black, T. L., Deaven, D. G., DiMego, G. J., Zhao, Q., Baldwin, M., Junker, N. W., Lin, Y., 1996: Changes to the Operational "Early" Eta  Analysis/Forecast System at the 
National Centers for Environmental Prediction. \wf \vol 11 391-416.

\ref Gallus Jr., W. A., 1999: Eta Simulations of Three Extreme Precipitation Events: Sensitivity to Resolution and Convective Parameterization. \wf \vol 14 405-426.

\ref Warner, Thomas T., Peterson, Ralph A., Treadon, Russell E., 1997:  A tutorial on lateral boundary conditions as a basic limitation to regional numerical weather prediction. {\it B.A.M.S. } \vol 78 2599-2617.




\end{multicols}


\end{document}
